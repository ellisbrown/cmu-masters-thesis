% writing
\newcommand{\eg}{{\it e.g.}\xspace}
\newcommand{\ie}{{\it i.e.}\xspace}
\newcommand{\cf}{{\it cf.}\xspace}
\newcommand{\etc}{{\it etc.}\xspace}
\newcommand{\wrt}{{\it w.r.t.}\xspace}

% std math stuff
\DeclareMathOperator*{\argmax}{argmax}
\DeclareMathOperator*{\argmin}{argmin}
\DeclareMathOperator*{\diag}{diag} \DeclareMathOperator*{\tr}{tr}
\DeclareMathOperator*{\maximize}{maximize}
\DeclareMathOperator*{\minimize}{minimize}
\DeclareMathOperator*{\st}{s.t.}
\DeclareMathOperator*{\subjectto}{subject\;to}
\DeclareMathOperator*{\vect}{vec} \DeclareMathOperator*{\mat}{mat}
\DeclareMathOperator{\prox}{prox}

\newcommand{\I}{\mathcal{I}}
\newcommand{\J}{\mathcal{J}}
\newcommand{\RR}{\mathbb{R}}
\newcommand{\R}{\mathbb{R}}
\newcommand{\dd}{\mathsf{d}}
\newcommand{\DD}{\mathsf{D}}

\newcommand{\reals}{{\mbox{\bf R}}}
\newcommand{\integers}{{\mbox{\bf Z}}}
\newcommand{\symm}{{\mbox{\bf S}}}  % symmetric matrices

% \newcommand{\diag}{\mathop{\bf diag}}
% \newcommand{\argmax}{\mathop{\rm argmax}}
% \newcommand{\argmin}{\mathop{\rm argmin}}
% \newcommand{\prox}{{\bf prox}}
\newcommand{\Range}{\mbox{\textrm{range}}}
\newcommand{\Nullspace}{\mbox{\textrm{nullspace}}}
\newcommand{\range}{{\mathcal{R}}}
\newcommand{\nullspace}{{\mathcal{N}}}
\newcommand{\Rank}{\mathop{\bf Rank}}
\newcommand{\Tr}{\mathop{\bf Tr}}
\newcommand{\Expect}{\mathop{\bf E{}}}
\newcommand{\Prob}{\mathop{\bf Prob}}
\newcommand{\var}{\mathop{\bf var}}
\newcommand{\sign}{\mathop{\bf sign}}
\newcommand{\card}{\mathop{\bf card}}
\newcommand{\dist}{\mathbf{dist}}
\newcommand{\ones}{\mathbf 1}
\newcommand{\length}{\mathbf{length}}
\newcommand{\dom}{\mathop{\bf dom}}
\newcommand{\env}{\mathop{\mathbf{env}}}

\newcommand{\mX}{\mathcal X}
\newcommand{\sample}{\mathop{\bf sample}}

% matrices
\newcommand{\EA}{\end{array}}
\newcommand{\BA}{\begin{array}}


% color macros

\newcommand{\green}[1]{\textcolor{ForestGreen}{#1}}
\newcommand{\red}[1]{\textcolor{red}{#1}}

\newcommand{\hred}[1]{{\color{red}#1}}
\newcommand{\hgreen}[1]{{\color{green!60!black}#1}}

% flag to show changes or not
\newif\ifchanges
\changestrue
% \changesfalse

\newcommand{\newchange}[1]{
    \ifchanges
        \green{#1}
    \else
        #1
    \fi
}

\newcommand{\newdelete}[1]{
    \ifchanges
        \red{#1}
    \else
    \fi
}



% pretty print numbers
% https://tex.stackexchange.com/a/6130
\newcount\ppnum
\newcommand\ppnumber[1]{%
        \ppnum=#1\relax
        \ifnum\ppnum<0
                $-$%
                \ppnum=-\ppnum
        \fi
        \let\pptemp\empty
        \loop\ifnum\ppnum>999
                \count255=\ppnum
                \divide\ppnum by1000
                \count255=\numexpr \count255 - 1000*\ppnum \relax
                \edef\pptemp{,\ifnum\count255<100 0\ifnum\count255<10 0\fi\fi
                             \the\count255 \pptemp}%
        \repeat
        \the\ppnum
        \pptemp
}
